% **************************************************************************
%
%  Package 'caratula', version 0.5 (para componer caratulas de TPs del DC).
%
%  En caso de dudas, problemas o sugerencias sobre este package escribir a
%  Brian J. Cardiff (bcardif arroba gmail.com).
%  Nico Rosner (nrosner arroba dc.uba.ar).
%
% **************************************************************************

% ----- Informacion sobre el package para el sistema -----------------------

\NeedsTeXFormat{LaTeX2e}
\ProvidesPackage{caratula}[2013/08/04 v0.5 Para componer caratulas de TPs del DC]
\RequirePackage{ifthen}
\usepackage[pdftex]{graphicx}

% ----- Imprimir un mensajito al procesar un .tex que use este package -----

\typeout{Cargando package 'caratula' v0.5 (2013/08/04)}

% ----- Algunas variables --------------------------------------------------

\let\Materia\relax
\let\Submateria\relax
\let\Titulo\relax
\let\Subtitulo\relax
\let\Grupo\relax
\let\Fecha\relax
\let\Logoimagefile\relax
\newcommand{\LabelIntegrantes}{}
\newboolean{showLU}
\newboolean{showEntregas}
\newboolean{showDirectores}
\newboolean{showCoDirectores}

% ----- Comandos para que el usuario defina las variables ------------------

\def\materia#1{\def\Materia{#1}}
\def\submateria#1{\def\Submateria{#1}}
\def\titulo#1{\def\Titulo{#1}}
\def\subtitulo#1{\def\Subtitulo{#1}}
\def\grupo#1{\def\Grupo{#1}}
\def\fecha#1{\def\Fecha{#1}}
\def\logoimagefile#1{\def\Logoimagefile{#1}}

% ----- Token list para los integrantes ------------------------------------

\newtoks\intlist\intlist={}

\newtoks\intlistSinLU\intlistSinLU={}

\newcounter{integrantesCount}
\setcounter{integrantesCount}{0}
\newtoks\intTabNombre\intTabNombre={}
\newtoks\intTabLU\intTabLU={}
\newtoks\intTabEmail\intTabEmail={}

\newcounter{directoresCount}
\setcounter{directoresCount}{0}
\newtoks\direcTabNombre\direcTabNombre={}
\newtoks\direcTabEmail\direcTabEmail={}

\newcounter{coDirectoresCount}
\setcounter{coDirectoresCount}{0}
\newtoks\codirecTabNombre\codirecTabNombre={}
\newtoks\codirecTabEmail\codirecTabEmail={}


% ----- Comando para que el usuario agregue integrantes --------------------

\def\integrante#1#2#3{%
    \intlist=\expandafter{\the\intlist\rule{0pt}{1.2em}#1&#2&\tt #3\\[0.2em]}%
    \intlistSinLU=\expandafter{\the\intlistSinLU\rule{0pt}{1.2em}#1 & \tt #3\\[0.2em]}%
    %
    \ifthenelse{\value{integrantesCount} > 0}{%
        \intTabNombre=\expandafter{\the\intTabNombre & #1}%
        \intTabLU=\expandafter{\the\intTabLU & #2}%
        \intTabEmail=\expandafter{\the\intTabEmail & \tt #3}%
    }{
        \intTabNombre=\expandafter{\the\intTabNombre #1}%
        \intTabLU=\expandafter{\the\intTabLU #2}%
        \intTabEmail=\expandafter{\the\intTabEmail \tt #3}%
    }%
    \addtocounter{integrantesCount}{1}%
}

\def\director#1#2{%
    \ifthenelse{\value{directoresCount} > 0}{%
        \direcTabNombre=\expandafter{\the\direcTabNombre & #1}%
        \direcTabEmail=\expandafter{\the\direcTabEmail & \tt #2}%
    }{
        \direcTabNombre=\expandafter{\the\direcTabNombre #1}%
        \direcTabEmail=\expandafter{\the\direcTabEmail \tt #2}%
    }%
    \addtocounter{directoresCount}{1}%
}

\def\codirector#1#2{%
    \ifthenelse{\value{coDirectoresCount} > 0}{%
        \codirecTabNombre=\expandafter{\the\codirecTabNombre & #1}%
        \codirecTabEmail=\expandafter{\the\codirecTabEmail & \tt #2}%
    }{
        \codirecTabNombre=\expandafter{\the\codirecTabNombre #1}%
        \codirecTabEmail=\expandafter{\the\codirecTabEmail \tt #2}%
    }%
    \addtocounter{coDirectoresCount}{1}%
}


% ----- Macro para generar la tabla de integrantes -------------------------

\newcommand{\tablaIntegrantes}{\ }

\newcommand{\tablaIntegrantesVertical}{%
\ifthenelse{\boolean{showLU}}{%
    \begin{tabular}[t]{| l @{\hspace{4ex}} c @{\hspace{4ex}} l|}
        \hline
        \multicolumn{1}{|c}{\rule{0pt}{1.2em} \LabelIntegrantes} & LU &  \multicolumn{1}{c|}{Correo electr\'onico} \\[0.2em]
        \hline \hline
        \the\intlist
        \hline
    \end{tabular}
}{
    \begin{tabular}[t]{| l @{\hspace{4ex}} @{\hspace{4ex}} l|}
        \hline
        \multicolumn{1}{|c}{\rule{0pt}{1.2em} \LabelIntegrantes} &  \multicolumn{1}{c|}{Correo electr\'onico} \\[0.2em]
        \hline \hline
        \the\intlistSinLU
        \hline
    \end{tabular}
    }%
}

\newcommand{\tablaIntegrantesHorizontal}{%
    \begin{tabular}[t]{ *{\value{integrantesCount}}{c} }
    \the\intTabNombre \\%
\ifthenelse{\boolean{showLU}}{
    \the\intTabLU \\%
}{}
    \the\intTabEmail %
    \end{tabular}%
}

\newcommand{\tablaDirectores}{%
\ifthenelse{\boolean{showDirectores}}{%
	\bigskip
	Directores

	\smallskip
    \begin{tabular}[t]{ *{\value{directoresCount}}{c} }
    \the\direcTabNombre \\%
    \the\direcTabEmail %
    \end{tabular}%
}{}%
}

\newcommand{\tablaCoDirectores}{%
\ifthenelse{\boolean{showCoDirectores}}{%
	\bigskip
	Co-Directores

	\smallskip
    \begin{tabular}[t]{ *{\value{coDirectoresCount}}{c} }
    \the\codirecTabNombre \\%
    \the\codirecTabEmail %
    \end{tabular}%
}{}%
}

\newcommand{\tablaEntregas}{%
\ifthenelse{\boolean{showEntregas}}{%
  \bigskip%
  \begin{tabular}[t]{|l p{3.5cm} p{1.5cm}|}%
  \hline%
  \rule{0pt}{1.2em} Instancia & Docente & Nota \\[0.2em] %
  \hline%
  \hline%
  \rule{0pt}{1.2em} Primera entrega & & \\[0.2em] %
  \hline%
  \rule{0pt}{1.2em} Segunda entrega & & \\[0.2em] %
  \hline%
  \end{tabular}%
}{}%
}

% ----- Codigo para manejo de errores --------------------------------------

\def\se{\let\ifsetuperror\iftrue}
\def\ifsetuperror{%
    \let\ifsetuperror\iffalse
    \ifx\Materia\relax\se\errhelp={Te olvidaste de proveer una \materia{}.}\fi
    \ifx\Titulo\relax\se\errhelp={Te olvidaste de proveer un \titulo{}.}\fi
    \edef\mlist{\the\intlist}\ifx\mlist\empty\se%
    \errhelp={Tenes que proveer al menos un \integrante{nombre}{lu}{email}.}\fi
    \expandafter\ifsetuperror}

\def\aftermaketitle{%
  \setcounter{page}{1}
}

% ----- \maketitletxt correspondiente a la versi�n v0.2.1 (texto v0.2 + fecha ) ---------

\def\maketitletxt{%
    \ifsetuperror\errmessage{Faltan datos de la caratula! Ingresar 'h' para mas informacion.}\fi
    \thispagestyle{empty}
    \begin{center}
    \vspace*{\stretch{2}}
    {\LARGE\textbf{\Materia}}\\[1em]
    \ifx\Submateria\relax\else{\Large \Submateria}\\[0.5em]\fi
    \ifx\Fecha\relax\else{\Large \Fecha}\\[0.5em]\fi
    \par\vspace{\stretch{1}}
    {\large Departamento de Computaci\'on}\\[0.5em]
    {\large Facultad de Ciencias Exactas y Naturales}\\[0.5em]
    {\large Universidad de Buenos Aires}
    \par\vspace{\stretch{3}}
    {\Large \textbf{\Titulo}}\\[0.8em]
    {\Large \Subtitulo}
    \par\vspace{\stretch{3}}
    \ifx\Grupo\relax\else\textbf{\Grupo}\par\bigskip\fi
    \tablaIntegrantes
    \end{center}
    \vspace*{\stretch{3}}
    \newpage\aftermaketitle}

% ----- \maketitletxtlogo correspondiente v0.2.1 (texto con fecha y logo) ---------

\def\maketitletxtlogo{%
    \ifsetuperror\errmessage{Faltan datos de la caratula! Ingresar 'h' para mas informacion.}\fi
    \thispagestyle{empty}
    \begin{center}
    \ifx\Logoimagefile\relax\else\includegraphics{\Logoimagefile}\fi \hfill \includegraphics{logo_dc.jpg}\\[1em]
    \vspace*{\stretch{2}}
    {\LARGE\textbf{\Materia}}\\[1em]
    \ifx\Submateria\relax\else{\Large \Submateria}\\[0.5em]\fi
    \ifx\Fecha\relax\else{\large \Fecha}\\[0.5em]\fi
    \par\vspace{\stretch{1}}
    {\large Departamento de Computaci\'on}\\[0.5em]
    {\large Facultad de Ciencias Exactas y Naturales}\\[0.5em]
    {\large Universidad de Buenos Aires}
    \par\vspace{\stretch{3}}
    {\Large \textbf{\Titulo}}\\[0.8em]
    {\Large \Subtitulo}
    \par\vspace{\stretch{3}}
    \ifx\Grupo\relax\else\textbf{\Grupo}\par\bigskip\fi
    \tablaIntegrantes
    \end{center}
    \vspace*{\stretch{4}}
    \newpage\aftermaketitle}

% ----- \maketitlegraf correspondiente a la versi�n v0.3 (gr�fica) -------------

\def\maketitlegraf{%
    \ifsetuperror\errmessage{Faltan datos de la caratula! Ingresar 'h' para mas informacion.}\fi
%
    \thispagestyle{empty}

    \ifx\Logoimagefile\relax\else\includegraphics{\Logoimagefile}\fi \hfill \includegraphics{logo_dc.jpg}

    \vspace*{.06 \textheight}

    \noindent \textbf{\huge \Titulo}  \medskip \\
    \ifx\Subtitulo\relax\else\noindent\textbf{\large \Subtitulo} \\ \fi%
    \noindent \rule{\textwidth}{1 pt}

    {\noindent\large\Fecha \hspace*\fill \Materia} \\
    \ifx\Submateria\relax\else{\noindent \hspace*\fill \Submateria}\fi%

    \medskip%
    \begin{center}
        \ifx\Grupo\relax\else\textbf{\Grupo}\par\bigskip\fi
        \tablaIntegrantes

        \tablaDirectores

        \tablaCoDirectores

        \tablaEntregas
    \end{center}%
    \vfill%
%
    \begin{minipage}[t]{\textwidth}
        \begin{minipage}[t]{.55 \textwidth}
            \includegraphics{logo_uba.jpg}
        \end{minipage}%%
        \begin{minipage}[b]{.45 \textwidth}
            \textbf{\textsf{Facultad de Ciencias Exactas y Naturales}} \\
            \textsf{Universidad de Buenos Aires} \\
            {\scriptsize %
            Ciudad Universitaria - (Pabell\'on I/Planta Baja) \\
                Intendente G\"uiraldes 2610 - C1428EGA \\
            Ciudad Aut\'onoma de Buenos Aires - Rep. Argentina \\
                Tel/Fax: (++54 +11) 4576-3300 \\
            http://www.exactas.uba.ar \\
            }
        \end{minipage}
    \end{minipage}%
%
    \newpage\aftermaketitle}

% ----- Reemplazamos el comando \maketitle de LaTeX con el nuestro ---------
\renewcommand{\maketitle}{\maketitlegraf}

% ----- Dependiendo de las opciones ---------
%
% opciones:
%   txt     : caratula solo texto.
%   txtlogo : caratula txt con logo del DC y del grupo (opcional).
%   graf    : (default) caratula grafica con logo del DC, UBA y del grupo (opcional).
%
\@makeother\*% some package redefined it as a letter (as color.sty)
%
% Layout general de la caratula
%
\DeclareOption{txt}{\renewcommand{\maketitle}{\maketitletxt}}
\DeclareOption{txtlogo}{\renewcommand{\maketitle}{\maketitletxtlogo}}
\DeclareOption{graf}{\renewcommand{\maketitle}{\maketitlegraf}}
%
% Etiqueta Autores o Integrantes
%
\DeclareOption{integrante}{\renewcommand{\LabelIntegrantes}{Integrante}}
\DeclareOption{autor}{\renewcommand{\LabelIntegrantes}{Autor}}
%
% Formato tabla de integrantes
%
\DeclareOption{intVert}{\renewcommand{\tablaIntegrantes}{\tablaIntegrantesVertical}}
\DeclareOption{intHoriz}{\renewcommand{\tablaIntegrantes}{\tablaIntegrantesHorizontal}}
\DeclareOption{conLU}{\setboolean{showLU}{true}}
\DeclareOption{sinLU}{\setboolean{showLU}{false}}
\DeclareOption{conEntregas}{\setboolean{showEntregas}{true}}
\DeclareOption{sinEntregas}{\setboolean{showEntregas}{false}}
\DeclareOption{showDirectores}{\setboolean{showDirectores}{true}}
\DeclareOption{hideDirectores}{\setboolean{showDirectores}{false}}
\DeclareOption{showCoDirectores}{\setboolean{showCoDirectores}{true}}
\DeclareOption{hideCoDirectores}{\setboolean{showCoDirectores}{false}}
%
% Opciones predeterminadas
%
\ExecuteOptions{intVert}%
\ExecuteOptions{graf}%
\ExecuteOptions{integrante}%
\ExecuteOptions{conLU}%
\ExecuteOptions{hideDirectores}%
\ExecuteOptions{hideCoDirectores}%
\ExecuteOptions{sinEntregas}%
%
\ProcessOptions\relax
